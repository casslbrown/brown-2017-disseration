\documentclass[english,man]{apa6}

\usepackage{amssymb,amsmath}
\usepackage{ifxetex,ifluatex}
\usepackage{fixltx2e} % provides \textsubscript
\ifnum 0\ifxetex 1\fi\ifluatex 1\fi=0 % if pdftex
  \usepackage[T1]{fontenc}
  \usepackage[utf8]{inputenc}
\else % if luatex or xelatex
  \ifxetex
    \usepackage{mathspec}
    \usepackage{xltxtra,xunicode}
  \else
    \usepackage{fontspec}
  \fi
  \defaultfontfeatures{Mapping=tex-text,Scale=MatchLowercase}
  \newcommand{\euro}{€}
\fi
% use upquote if available, for straight quotes in verbatim environments
\IfFileExists{upquote.sty}{\usepackage{upquote}}{}
% use microtype if available
\IfFileExists{microtype.sty}{\usepackage{microtype}}{}
\usepackage{color}
\usepackage{fancyvrb}
\newcommand{\VerbBar}{|}
\newcommand{\VERB}{\Verb[commandchars=\\\{\}]}
\DefineVerbatimEnvironment{Highlighting}{Verbatim}{commandchars=\\\{\}}
% Add ',fontsize=\small' for more characters per line
\usepackage{framed}
\definecolor{shadecolor}{RGB}{248,248,248}
\newenvironment{Shaded}{\begin{snugshade}}{\end{snugshade}}
\newcommand{\KeywordTok}[1]{\textcolor[rgb]{0.13,0.29,0.53}{\textbf{{#1}}}}
\newcommand{\DataTypeTok}[1]{\textcolor[rgb]{0.13,0.29,0.53}{{#1}}}
\newcommand{\DecValTok}[1]{\textcolor[rgb]{0.00,0.00,0.81}{{#1}}}
\newcommand{\BaseNTok}[1]{\textcolor[rgb]{0.00,0.00,0.81}{{#1}}}
\newcommand{\FloatTok}[1]{\textcolor[rgb]{0.00,0.00,0.81}{{#1}}}
\newcommand{\ConstantTok}[1]{\textcolor[rgb]{0.00,0.00,0.00}{{#1}}}
\newcommand{\CharTok}[1]{\textcolor[rgb]{0.31,0.60,0.02}{{#1}}}
\newcommand{\SpecialCharTok}[1]{\textcolor[rgb]{0.00,0.00,0.00}{{#1}}}
\newcommand{\StringTok}[1]{\textcolor[rgb]{0.31,0.60,0.02}{{#1}}}
\newcommand{\VerbatimStringTok}[1]{\textcolor[rgb]{0.31,0.60,0.02}{{#1}}}
\newcommand{\SpecialStringTok}[1]{\textcolor[rgb]{0.31,0.60,0.02}{{#1}}}
\newcommand{\ImportTok}[1]{{#1}}
\newcommand{\CommentTok}[1]{\textcolor[rgb]{0.56,0.35,0.01}{\textit{{#1}}}}
\newcommand{\DocumentationTok}[1]{\textcolor[rgb]{0.56,0.35,0.01}{\textbf{\textit{{#1}}}}}
\newcommand{\AnnotationTok}[1]{\textcolor[rgb]{0.56,0.35,0.01}{\textbf{\textit{{#1}}}}}
\newcommand{\CommentVarTok}[1]{\textcolor[rgb]{0.56,0.35,0.01}{\textbf{\textit{{#1}}}}}
\newcommand{\OtherTok}[1]{\textcolor[rgb]{0.56,0.35,0.01}{{#1}}}
\newcommand{\FunctionTok}[1]{\textcolor[rgb]{0.00,0.00,0.00}{{#1}}}
\newcommand{\VariableTok}[1]{\textcolor[rgb]{0.00,0.00,0.00}{{#1}}}
\newcommand{\ControlFlowTok}[1]{\textcolor[rgb]{0.13,0.29,0.53}{\textbf{{#1}}}}
\newcommand{\OperatorTok}[1]{\textcolor[rgb]{0.81,0.36,0.00}{\textbf{{#1}}}}
\newcommand{\BuiltInTok}[1]{{#1}}
\newcommand{\ExtensionTok}[1]{{#1}}
\newcommand{\PreprocessorTok}[1]{\textcolor[rgb]{0.56,0.35,0.01}{\textit{{#1}}}}
\newcommand{\AttributeTok}[1]{\textcolor[rgb]{0.77,0.63,0.00}{{#1}}}
\newcommand{\RegionMarkerTok}[1]{{#1}}
\newcommand{\InformationTok}[1]{\textcolor[rgb]{0.56,0.35,0.01}{\textbf{\textit{{#1}}}}}
\newcommand{\WarningTok}[1]{\textcolor[rgb]{0.56,0.35,0.01}{\textbf{\textit{{#1}}}}}
\newcommand{\AlertTok}[1]{\textcolor[rgb]{0.94,0.16,0.16}{{#1}}}
\newcommand{\ErrorTok}[1]{\textcolor[rgb]{0.64,0.00,0.00}{\textbf{{#1}}}}
\newcommand{\NormalTok}[1]{{#1}}

% Table formatting
\usepackage{longtable, booktabs}
\usepackage{lscape}
% \usepackage[counterclockwise]{rotating}   % Landscape page setup for large tables
\usepackage{multirow}		% Table styling
\usepackage{tabularx}		% Control Column width
\usepackage[flushleft]{threeparttable}	% Allows for three part tables with a specified notes section
\usepackage{threeparttablex}            % Lets threeparttable work with longtable

% Create new environments so endfloat can handle them
% \newenvironment{ltable}
%   {\begin{landscape}\begin{center}\begin{threeparttable}}
%   {\end{threeparttable}\end{center}\end{landscape}}

\newenvironment{lltable}
  {\begin{landscape}\begin{center}\begin{ThreePartTable}}
  {\end{ThreePartTable}\end{center}\end{landscape}}

  \usepackage{ifthen} % Only add declarations when endfloat package is loaded
  \ifthenelse{\equal{\string man}{\string man}}{%
   \DeclareDelayedFloatFlavor{ThreePartTable}{table} % Make endfloat play with longtable
   % \DeclareDelayedFloatFlavor{ltable}{table} % Make endfloat play with lscape
   \DeclareDelayedFloatFlavor{lltable}{table} % Make endfloat play with lscape & longtable
  }{}%



% The following enables adjusting longtable caption width to table width
% Solution found at http://golatex.de/longtable-mit-caption-so-breit-wie-die-tabelle-t15767.html
\makeatletter
\newcommand\LastLTentrywidth{1em}
\newlength\longtablewidth
\setlength{\longtablewidth}{1in}
\newcommand\getlongtablewidth{%
 \begingroup
  \ifcsname LT@\roman{LT@tables}\endcsname
  \global\longtablewidth=0pt
  \renewcommand\LT@entry[2]{\global\advance\longtablewidth by ##2\relax\gdef\LastLTentrywidth{##2}}%
  \@nameuse{LT@\roman{LT@tables}}%
  \fi
\endgroup}


\ifxetex
  \usepackage[setpagesize=false, % page size defined by xetex
              unicode=false, % unicode breaks when used with xetex
              xetex]{hyperref}
\else
  \usepackage[unicode=true]{hyperref}
\fi
\hypersetup{breaklinks=true,
            pdfauthor={},
            pdftitle={The title},
            colorlinks=true,
            citecolor=blue,
            urlcolor=blue,
            linkcolor=black,
            pdfborder={0 0 0}}
\urlstyle{same}  % don't use monospace font for urls

\setlength{\parindent}{0pt}
%\setlength{\parskip}{0pt plus 0pt minus 0pt}

\setlength{\emergencystretch}{3em}  % prevent overfull lines

\ifxetex
  \usepackage{polyglossia}
  \setmainlanguage{}
\else
  \usepackage[english]{babel}
\fi

% Manuscript styling
\captionsetup{font=singlespacing,justification=justified}
\usepackage{csquotes}
\usepackage{upgreek}

 % Line numbering
  \usepackage{lineno}
  \linenumbers


\usepackage{tikz} % Variable definition to generate author note

% fix for \tightlist problem in pandoc 1.14
\providecommand{\tightlist}{%
  \setlength{\itemsep}{0pt}\setlength{\parskip}{0pt}}

% Essential manuscript parts
  \title{The title}

  \shorttitle{Title}


  \author{Cassandra Brown\textsuperscript{1}~\& Second Author\textsuperscript{1,2}}

  \def\affdep{{"", ""}}%
  \def\affcity{{"", ""}}%

  \affiliation{
    \vspace{0.5cm}
          \textsuperscript{1} University of Victoria\\
          \textsuperscript{2}   }

  \authornote{
    \newcounter{author}
    Complete departmental affiliations for each author (note the
    indentation, if you start a new paragraph). Enter author note here.

                      Correspondence concerning this article should be addressed to Cassandra Brown, Postal address. E-mail: \href{mailto:clb@uvic.ca}{\nolinkurl{clb@uvic.ca}}
                          }


  \abstract{Enter abstract here (note the indentation, if you start a new
paragraph).}
  \keywords{keywords \\

    \indent Word count: X
  }





\usepackage{amsthm}
\newtheorem{theorem}{Theorem}
\newtheorem{lemma}{Lemma}
\theoremstyle{definition}
\newtheorem{definition}{Definition}
\newtheorem{corollary}{Corollary}
\newtheorem{proposition}{Proposition}
\theoremstyle{definition}
\newtheorem{example}{Example}
\theoremstyle{remark}
\newtheorem*{remark}{Remark}
\begin{document}

\maketitle

\setcounter{secnumdepth}{0}



\begin{Shaded}
\begin{Highlighting}[]
\KeywordTok{getwd}\NormalTok{()}
\end{Highlighting}
\end{Shaded}

\begin{verbatim}
## [1] "/Users/cassandrabrown/Github/brown-2017-disseration"
\end{verbatim}

\section{Methods}\label{methods}

We report how we determined our sample size, all data exclusions (if
any), all manipulations, and all measures in the study.

\subsection{Participants}\label{participants}

\subsection{Material}\label{material}

\subsection{Procedure}\label{procedure}

\subsection{Data analysis}\label{data-analysis}

\section{Results}\label{results}

\begin{table}[tbp]
\begin{center}
\begin{threeparttable}
\caption{\label{tab:descriptive-statistics}Descriptive statistics by year}
\begin{tabular}{lllllll}
\toprule
 & \multicolumn{1}{c}{2004} & \multicolumn{1}{c}{2006} & \multicolumn{1}{c}{2008} & \multicolumn{1}{c}{2010} & \multicolumn{1}{c}{2012} & \multicolumn{1}{c}{2014}\\
\midrule
 & M (SD) & M (SD) & M (SD) & M (SD) & M (SD) & M (SD)\\
 & n = 5531 & n = 5720 & n = 5810 & n = 5698 & n = 5165 & n = 4454\\
Women (\%) & 59.68 & 51.21 & 50 & 50 & 50 & 50\\
Age & 72.13 (5.82) & 74.07 (5.83) & 76.01 (5.85) & 78.31 (5.77) & 79.62 (5.47) & 81.15 (5.26)\\
Yrs Education & 12.38 (3.1) & 12.38 (3.1) & 12.38 (3.1) & 12.38 (3.1) & 12.38 (3.1) & 12.38 (3.1)\\
Health Conditions & 1.96 (1.19) & 2.13 (1.22) & 2.28 (1.23) & 2.47 (1.25) & 2.53 (1.26) & 2.58 (1.26)\\
Mental status & 8.53 (0.79) & 8.51 (0.81) & 8.44 (0.88) & 8.07 (1.1) & 8.09 (1.16) & 7.97 (1.32)\\
Word recall immediate & 5.45 (1.5) & 5.31 (1.53) & 5.18 (1.54) & 4.86 (1.64) & 4.74 (1.63) & 4.63 (1.65)\\
Word recall delayed & 4.4 (1.84) & 4.23 (1.89) & 4.12 (1.88) & 3.75 (1.95) & 3.61 (1.97) & 3.49 (1.96)\\
Psychosocial Variables & n = 1061 & n = 2787 & n = 2737 & n = 2646 & n = 2235 & n = 2031\\
Loneliness & 1.35 (0.47) & 1.43 (0.51) & 1.44 (0.51) & 1.43 (0.51) & 1.46 (0.5) & 1.43 (0.51)\\
Social contact & 30.56 (8.37) & 29.6 (8.14) & 29.61 (8.6) & 29.41 (8.49) & 28.99 (8.78) & 28.65 (8.75)\\
Social support & 9.81 (1.53) & 9.58 (1.52) & 9.56 (1.6) & 9.58 (1.56) & 9.61 (1.59) & 9.58 (1.61)\\
Depression & 1.16 (1.72) & 1.23 (1.75) & 1.23 (1.73) & 1.3 (1.78) & 1.34 (1.82) & 1.38 (1.86)\\
Social network & 3.44 (0.76) & 3.41 (0.72) & 3.31 (0.77) & 3.25 (0.79) & 3.12 (0.84) & 3.04 (0.86)\\
\bottomrule
\end{tabular}
\end{threeparttable}
\end{center}
\end{table}

\section{Discussion}\label{discussion}

\newpage

\section{References}\label{references}

\setlength{\parindent}{-0.5in} \setlength{\leftskip}{0.5in}

\newpage

\label{tab:immediate-word-recall-model-summaries}\emph{Model Fit Indices for
Immediate Word Recall}

\begin{longtable}[]{@{}lrrlrrrrrr@{}}
\toprule
Model & \(\chi^2\) & df & CM & \(\Delta\chi^2\) & df\(\Delta\) & CFI &
TLI & RMSEA & SRMR\tabularnewline
\midrule
\endhead
Autoregressive, univariate & 2803.192 & 10 & - & NA & NA & 0.631 & 0.446
& 0.217 & 0.231\tabularnewline
LGM & 112.905 & 16 & - & NA & NA & 0.987 & 0.988 & 0.032 &
0.014\tabularnewline
LGM, quadratic & 102.364 & 15 & - & NA & NA & 0.988 & 0.988 & 0.031 &
0.014\tabularnewline
ALT, full model & 64.897 & 11 & 4 & NaN & 0 & 0.993 & 0.990 & 0.029 &
0.015\tabularnewline
LGM, nested in ALT & 112.905 & 16 & 4 & 48.03457 & 5 & 0.987 & 0.988 &
0.032 & 0.014\tabularnewline
ALT, no slope variance & 115.864 & 13 & 4 & 49.04157 & 2 & 0.986 & 0.984
& 0.036 & 0.021\tabularnewline
ALT, no slope & 646.224 & 14 & 4 & 538.75120 & 3 & 0.916 & 0.910 & 0.087
& 0.058\tabularnewline
ALT, fixed regressions & 99.613 & 15 & 4 & 34.66512 & 4 & 0.989 & 0.989
& 0.031 & 0.014\tabularnewline
\bottomrule
\end{longtable}

\section{Supplemental Material}\label{supplemental-material}

The results from the univariate models of each cognitive and social
process are presented in the supplemental table. For word recall
immediate, the model comparison results show that the autoregressive
model has an inadequate fit to the data. The latent growth model (LGM)
and the ALT full model have comparable fit indices to the data. However,
the LGM nested within the ALT model shows significantly poorer fit
according the ∆χ2 statistic. Additional restrictions, constraining the
slope variance to 0 (model 5), eliminating the slope term (model 6), and
constraining the autoregressive parameters to be the same over time
(model 7) resulted in significantly poorer model fit. Immediate word
recall performance decreased over time (\(\beta\) = , \emph{p} ).

\begin{verbatim}
## [1] "working?"
\end{verbatim}

\(\rho_{21}\)






\end{document}
